\documentclass[a4paper,10pt]{article}
\topmargin-1cm
\addtolength{\textheight}{2.5cm}
\addtolength{\textwidth}{2cm}
\usepackage{times}

%Allow Colors
\usepackage[usenames, dvipsnames]{color}
%Allow overbrace
\usepackage{amsmath}
%Allow XOR
\usepackage{amssymb}

\usepackage{verbatim}
\usepackage{graphicx}
\usepackage[german]{babel}
\usepackage[latin1]{inputenc}

\setlength{\parindent}{0cm}

\title{Latex Template}
\author{David Herzig}
\date{WS2004}

\def\doubleunderline#1{\underline{\underline{#1}}}

\begin{document}

HF-ICT - H"ohere Fachschule f"ur Informations- und Kommunikationstechnologie\\
CH 4132 Muttenz\\
P.S"utterlin

\vspace{2mm}

\begin{center}
{\Large \bf Vorkurs Informatik}\\
Exercise sheet 3 - 2018 Edition
\end{center}

\vspace{2mm}

\line(1,0){400}

\vspace{5mm}

\section{Logik 1}
Gegeben:\\
\emph{es ist anstrengend}: a\\
\emph{es macht Spass}: b\\
Erstellen Sie Ausdr"ucke f"ur: \\
	 {\color{ForestGreen}
		 Ausdr"ucke sind nicht objektiv messbar !!!
	 }

\begin{enumerate}
\item Es ist anstrengend und trotzdem macht es Spass. \\
	 {\color{ForestGreen}
		 $a \land b$
	 }
\item Es macht keinen Spass wenn es anstrengend ist. \\
	 {\color{ForestGreen}
		 $a \land \lnot b$
	 }
\item Es ist nicht anstrengend, trotzdem macht es keinen Spass. \\
	 {\color{ForestGreen}
		 $\lnot a \land \lnot b$
	 }
\end{enumerate}

\section{Logik 2}
Gegeben:\\
p, q und r: WAHR\\
s: FALSCH\\
Ermitteln Sie den Wahrheitswert der folgenden Aussagen:

\begin{enumerate}
\item $p \land q \lor s$ \\ \\
	 {\color{ForestGreen}
		 $1 \land 1 \lor 0 =$ \\
		 $1 \lor 0 = $ \\
		 $\mathbf{\doubleunderline{WAHR}}$ 
	 }
\item $\lnot (p \land q) \lor s$ \\ \\
	 {\color{ForestGreen}
		 $\lnot (1 \land 1) \lor 0 =$ \\
		 $\lnot (1) \lor 0 =$ \\
		 $0 \lor 0 =$ \\
		 $\mathbf{\doubleunderline{FALSCH}}$ 
	 }
\item $\lnot p \lor \lnot r \lor \lnot s$ \\ \\
	 {\color{ForestGreen}
		 $\lnot 1 \lor \lnot 1 \lor \lnot 0 =$ \\
		 $0 \lor 0 \lor 1 =$ \\
		 $\mathbf{\doubleunderline{WAHR}}$ 
	 }
\end{enumerate}

\section{Boolsche Operationen}
F"uhren Sie die folgenden Operationen aus:

\begin{enumerate}
\item 12 and 3 \\ \\
	 {\color{ForestGreen}
		 1100 AND \\
		 0011 \\
		 ==== \\
		 0000 \\ \\
		 12 and 3 = $\mathbf{\doubleunderline{0}}$
	 }
\item 3 or 12 \\ \\
	 {\color{ForestGreen}
		 1100 OR \\
		 0011 \\
		 ==== \\
		 1111 \\ \\
		 3 or 12 = $\mathbf{\doubleunderline{15}}$
	 }
\item not 62 \\ \\
	 {\color{ForestGreen}
		 111110 NOT \\
		 ===== \\
		 000001 \\ \\
		 not 62 = $\mathbf{\doubleunderline{1}}$
	 }
\item 74 and 96 or 7 \\ \\
	 {\color{ForestGreen}
		 1001010 AND \\
		 1100000 \\
		 ====== \\
		 1000000 \\
		 \\
		 1000000 OR \\
		 0000111 \\
		 ====== \\
		 1000111 \\ \\
		 74 and 96 or 7 = $\mathbf{\doubleunderline{71}}$
	 }
\end{enumerate}

\section{Bitmaske}
In einer 1 Byte grossen Bitmaske lassen sich 8 einzelne ``Flags'' setzen, indem man jeweils die Bits mit den Wertigkeiten 1, 2, 4, 8, 16, 32, 64 und 128 mittels \textbf{OR} Operationen gezielt setzt. \\ \\
D.h. zum setzten der Flags \textbf{4} \textit{(WRITE\_ENABLED)}, \textbf{32} \textit{(DOUBLE\_SIZE)} und \textbf{128} \textit{(HIDDEN)} werden die drei Werte wie folgt verkn"upft: \\ \\
$4 \lor  32 \lor 128 = 164$ \\ \\
Die drei Bits sind dann im Byte wie folgt gesetzt: [10100100]. \\ \\
Wie k"onnen Sie nun nachpr"ufen ob in der Bitmaske die Flags \textit{(WRITE\_ENABLED)} und \textit{(HIDDEN)} gesetzt sind ? \\ \\
	 {\color{ForestGreen}
		\underline{Pr"ufen auf Flag 4}: \\ \\
		\textbf{(4 AND 164) $\mathbf{>}$ 0} \\ \\
		00000100 AND \\
		10100100 \\
		======= \\
		00000100 \\ \\ 
		4 AND 164 = $\mathbf{\doubleunderline{4}}$ \\
		
		\underline{Pr"ufen auf Flag 128}: \\ \\
		\textbf{(128 AND 164) $\mathbf{>}$ 0} \\ \\
		100000000 AND \\
		10100100 \\
		======= \\
		10000000 \\ \\ 
		128 AND 164 = $\mathbf{\doubleunderline{128}}$ \\
	 }


\section{Wahrheitstabelle}
Erstellen Sie f"ur den folgenden boolschen Ausdruck die Wahrheitstabelle:\\
$(\lnot p \land \lnot s) \lor \lnot(p \land \lnot s)$ \\
	 {\color{ForestGreen}
	 
	 	$Output = (\overbrace{\lnot p \land \lnot s}^\text{f1}) \lor \lnot ( \overbrace{p \land \lnot s}^\text{f2} )$ \\
	 
		\begin{tabular}{c|c|c|c|c}
		p & s & $f1=\lnot p \land \lnot s$ & $f2=p \land \lnot s$ & $Output = f1 \lor \lnot f2$ \\
		\hline
		0 & 0 & \textbf{1} & \textbf{0} & \textbf{1}\\
		0 & 1 & \textbf{0} & \textbf{0} & \textbf{1}\\
		1 & 0 & \textbf{0} & \textbf{1} & \textbf{0}\\
		1 & 1 & \textbf{0} & \textbf{0} & \textbf{1}\\
		\end{tabular}
		\\ \\
		\\
		\textbf{Weiterer L"osungsansatz (De Morgan):} \\
		$Output = \overbrace{(\lnot p \land \lnot s)}^\text{$\lnot(p \lor s)$} \lor \overbrace{\lnot ( p \land \lnot s}^\text{$(\lnot p \lor s)$} )$ \\
		$Output = \lnot(\overbrace{p \lor s}^\text{f1}) \lor \lnot p \lor s$ \\
		
		\begin{tabular}{c|c|c|c}
		p & s & $f1=p \lor s$ & $Output = \lnot f1 \lor \lnot p \lor s$ \\
		\hline
		0 & 0 & \textbf{0} & \textbf{1}\\
		0 & 1 & \textbf{1} & \textbf{1}\\
		1 & 0 & \textbf{1} & \textbf{0}\\
		1 & 1 & \textbf{1} & \textbf{1}\\
		\end{tabular}
	 }
\end{document}

