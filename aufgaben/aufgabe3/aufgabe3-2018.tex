\documentclass[a4paper,10pt]{article}
\topmargin-1cm
\addtolength{\textheight}{2.5cm}
\addtolength{\textwidth}{2cm}
\usepackage{times}

\usepackage{verbatim}
\usepackage{graphicx}
\usepackage[german]{babel}
\usepackage[latin1]{inputenc}

\setlength{\parindent}{0cm}

\title{Latex Template}
\author{David Herzig}
\date{WS2004}

\begin{document}

HF-ICT - H"ohere Fachschule f"ur Informations- und Kommunikationstechnologie\\
CH 4132 Muttenz\\
P.S"utterlin

\vspace{2mm}

\begin{center}
{\Large \bf Vorkurs Informatik}\\
Exercise sheet 2 - 2018 Edition
\end{center}

\vspace{2mm}

\line(1,0){400}

\vspace{5mm}

\section{Logik 1}
Gegeben:\\
\emph{es ist anstrengend}: a\\
\emph{es macht Spass}: b\\
Erstellen Sie Ausdr"ucke f"ur:

\begin{enumerate}
\item Es ist anstrengend und trotzdem macht es Spass.
\item Es macht keinen Spass wenn es anstrengend ist.
\item Es ist nicht anstrengend, trotzdem macht es keinen Spass.
\end{enumerate}

\section{Logik 2}
Gegeben:\\
p, q und r: WAHR\\
s: FALSCH\\
Ermitteln Sie den Wahrheitswert der folgenden Aussagen:

\begin{enumerate}
\item $p \land q \lor s$
\item $\lnot (p \land q) \lor s$
\item $\lnot p \lor \lnot r \lor \lnot s$
\end{enumerate}

\section{Boolsche Operationen}
F"uhren Sie die folgenden Operationen aus:

\begin{enumerate}
\item 12 and 3
\item 3 or 12
\item not 62
% 111111 XOR
% 111110
% =====
% 000001 Alle 1er werden 0er und umgekehrt
\item 74 and 96 or 7
% 1001010 AND
% 1100000
% =====
% 1000000 OR
% 0000111 
% =====
% 1000111

\end{enumerate}

\section{Bitmaske}
In einer 1 Byte grossen Bitmaske lassen sich 8 einzelne ``Flags'' setzen, indem man jeweils die Bits mit den Wertigkeiten 1, 2, 4, 8, 16, 32, 64 und 128 mittels \textbf{OR} Operationen gezielt setzt. \\ \\
D.h. zum setzten der Flags \textbf{4} \textit{(WRITE\_ENABLED)}, \textbf{32} \textit{(DOUBLE\_SIZE)} und \textbf{128} \textit{(HIDDEN)} werden die drei Werte wie folgt verkn"upft: \\ \\
$4 \lor  32 \lor 128 = 164$ \\ \\
Die drei Bits sind dann im Byte wie folgt gesetzt: [10100100]. \\ \\
Wie k"onnen Sie nun nachpr"ufen ob in der Bitmaske die Flags \textit{(WRITE\_ENABLED)} und \textit{(HIDDEN)} gesetzt sind ?
%Prüfen auf Flag 4 -> 4 AND 164 darf nicht 0 ergeben.%
%Prüfen auf Flag 128 -> 128 AND 164 darf nicht 0 ergeben.%

\section{Wahrheitstabelle}
Erstellen Sie f"ur den folgenden boolschen Ausdruck die Wahrheitstabelle:\\
$(\lnot p \land \lnot s) \lor \lnot(p \land \lnot s)$

\end{document}

