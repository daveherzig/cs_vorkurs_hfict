\documentclass[a4paper,10pt]{article}
\topmargin-1cm
\addtolength{\textheight}{2.5cm}
\addtolength{\textwidth}{2cm}
\usepackage{times}

\usepackage{verbatim}
\usepackage[dvips]{graphicx}
\usepackage[german]{babel}
\usepackage[latin1]{inputenc}

\setlength{\parindent}{0cm}

\title{Latex Template}
\author{David Herzig}
\date{WS2004}

\begin{document}

HF-ICT - H"ohere Fachschule f"ur Informations- und Kommunikationstechnologie\\
CH 4132 Muttenz\\
D.Herzig

\vspace{2mm}

\begin{center}
{\Large \bf Vorkurs Informatik}\\
Exercise sheet 3
\end{center}

\vspace{2mm}

\line(1,0){400}

\vspace{5mm}

\section{Logik 1}
Gegeben:\\
\emph{es regnet}: a\\
\emph{es ist kalt}: b\\
Erstellen Sie Ausdr"ucke f"ur:

\begin{enumerate}
\item Es regnet aber es ist nicht kalt
\item Wenn es regnet, so ist es kalt
\item Es stimmt nicht, dass es regnet oder kalt ist
\end{enumerate}

\section{Logik 2}
Gegeben:\\
p, q und r: WAHR\\
s: FALSCH\\
Ermitteln Sie den Wahrheitswert der folgenden Aussagen:

\begin{enumerate}
\item $p \land (q \lor s)$
\item $p \land q \lor s$
\item $\lnot p \lor r$
\end{enumerate}

\section{Boolsche Operationen}
F"uhren Sie die folgenden Operationen aus:

\begin{enumerate}
\item 14 and 25
\item 29 or 53
\item not 45
\item 74 and 34 or 56
\end{enumerate}

\section{Bit gesetzt?}
Die Zahl 56 ist in einer 8 Bit Zahl gespeichert. Wie kann gepr"uft werden,
ob das 3. Bit in dieser Zahl 1 resp. 0 ist?

\section{Wahrheitstabelle}
Erstellen Sie f"ur den folgenden boolschen Ausdruck die Wahrheitstabelle:\\
$(p \land q) \lor \lnot(p \land \lnot r)$

\end{document}

