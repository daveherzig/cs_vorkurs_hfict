\documentclass[a4paper,10pt]{article}
\topmargin-1cm
\addtolength{\textheight}{2.5cm}
\addtolength{\textwidth}{2cm}
\usepackage{times}

\usepackage{verbatim}
\usepackage[dvips]{graphicx}
\usepackage[german]{babel}
\usepackage[latin1]{inputenc}

\setlength{\parindent}{0cm}

\title{Latex Template}
\author{David Herzig}
\date{WS2003}

\begin{document}

KTSI - Kantonale Technikerschule f"ur Informatik\\
CH 4132 Muttenz\\
D.Herzig

\vspace{2mm}

\begin{center}
{\Large \bf Vorkurs Informatik}\\
Exercise sheet 5
\end{center}

\vspace{2mm}

\line(1,0){400}

\vspace{5mm}

\section{Daten"ubertragung}
Eine Festplatte hat pro Sektor 4096 Bits und besitzt 64 Sektoren pro Spur. Desweiteren hat die Festplatte eine Drehzahl von 7200UpM. Wie hoch ist die Daten"ubertragungsrate dieser Platte pro Spur?

\section{Laserdrucker}
In einer bestimmten Schriftart kann ein Monochrom Laserdrucker 50 Zeilen zu je 80 Zeichen pro Seite ausdrucken. Ein Zeichen belegt durchschnittlich ein K"astchen von 2x2 mm, wovon run 25\verb|%| mit Toner bedeckt sind und der Rest frei ist. Die Tonerschicht ist 25 $\mu$m dick. Die Tonerkassette des Druckers misst 25x8x2 cm. Wieviele Seiten k"onnen mit einer Tonerkassette bedruckt werden?

\section{Begriffe Magnetspeicher}
Erkl"aren Sie die Begriffe Spur, Sektor und Zylinder im Zusammenhang mit Festplatten.

\section{Daten"ubertragung}
Eine Firma er"offnet 10km von ihrem Hauptsitz ein zweites Nebengeb"aude.
Um die beiden Netzwerke miteinander zu verbinden stehen zwei M"oglichkeiten zur Verf"ugung:
Eine Kupferleitung (Kosten pro Meter 82.-) sowie eine Glasfaserleitung (Kosten pro Meter 211.-).
Ein Kupferkabel hat eine "ubertragungsrate von 10MBit/s und eine Glasfaserleitung von 100MBit/s.
Die Firma "ubertr"agt pro Tag eine Datenmenge von 14GB. Berechnen Sie in wieviel Prozent die
beiden M"oglichkeiten der "Ubertragung in Kosten und Geschwindigkeit auseinanderliegen.

\section{Zugriffszeit}
Angenommen, eine CPU hat einen Level 1 Cache und einen Level 2 Cache mit Zugriffszeiten
von 5 bzw. 10ns. Die RAM Zugriffszeit betr"agt 50ns. Angenommen, 20\verb|%|
der Zugriffe erfolgen auf Level 1 Cache, 60\verb|%| Zugriffe auf Level 2 Cache
und der Rest auf das RAM. Welche durchschnittliche Zugrisszeit ergibt sich?
Was w"are die Zugriffszeit, wenn der Computer keine Cache besitzen w"ure?


\end{document}

