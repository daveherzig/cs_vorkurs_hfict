\documentclass[a4paper,10pt]{article}
\topmargin-1cm
\addtolength{\textheight}{2.5cm}
\addtolength{\textwidth}{2cm}
\usepackage{times}

\usepackage{amsmath}
\usepackage[usenames, dvipsnames]{color}
\usepackage{verbatim}
\usepackage{graphicx}
\usepackage[german]{babel}
\usepackage[latin1]{inputenc}

\setlength{\parindent}{0cm}

\title{Latex Template}
\author{David Herzig}
\date{WS2003}

\def\doubleunderline#1{\underline{\underline{#1}}}

\begin{document}

HF-ICT - H"ohere Fachschule f"ur Informations- und Kommunikationstechnologie\\
CH 4132 Muttenz\\
P.S"utterlin

\vspace{2mm}

\begin{center}
{\Large \bf Vorkurs Informatik}\\
Exercise sheet 1 - 2018 Edition
\end{center}

\vspace{2mm}

\line(1,0){400}

\vspace{5mm}

\section{Zahlensysteme}

F"uhren Sie die folgenden Umwandlungen durch!

\begin{enumerate}
\item $11110000_{2} = ?_{10}$ \\    %240
	 {\color{ForestGreen}
		 $b=2$  \\
		 $P(b)= 0 \cdot 2^0 + 0 \cdot 2^1 + 0 \cdot 2^2 + 0 \cdot 2^3 + 1 \cdot 2^4+ 1 \cdot 2^5 + 1 \cdot 2^6 + 1 \cdot 2^7$ \\
		 $P(b)= 0 \cdot 1 + 0 \cdot 2 + 0 \cdot 4 + 0 \cdot 8 + 1 \cdot 16 + 1 \cdot 32 + 1 \cdot 64 + 1 \cdot 128$ \\
		 $P(b)= 16 + 32 + 64 + 128$ \\
		 $\mathbf{P(2)= \doubleunderline{240}}$
	 }
\item $2AA_{16} = ?_{2}$ \\               %1010101010
	 {\color{ForestGreen}
		 $\underbrace{10}_\text{$2_{16}$}\underbrace{1010}_\text{$A_{16}$}\underbrace{1010}_\text{$A_{16}$}$  \\
		 $\mathbf{?_{2} = \doubleunderline{1010101010}}$
	 }
\item $765_{8} = ?_{10}$ \\               %501
	 {\color{ForestGreen}
		 $b=8$  \\
		 $P(b)= 5 \cdot 8^0 + 6 \cdot 8^1 + 7 \cdot 8^2$ \\
		 $P(b)= 5 \cdot 1 + 6 \cdot 8 + 7 \cdot 64$ \\
		 $P(b)= 5 + 48 + 448$ \\
		 $\mathbf{P(8)= \doubleunderline{501}}$
	 }
\item $C0DE_{16} = ?_{2}$ \\            %1100000011011110
	 {\color{ForestGreen}
		 $\underbrace{1100}_\text{$C_{16}$}\underbrace{0000}_\text{$0_{16}$}\underbrace{1101}_\text{$D_{16}$}
		 \underbrace{1110}_\text{$E_{16}$}$  \\
		 $\mathbf{?_{2} = \doubleunderline{1100000011011110}}$
	 }
\item $53456_{10} = ?_{16}$ \\         %D0D0
	 {\color{ForestGreen}
		 $53456 : 16 = 3341 R=0$  \\
		 $3341 : 16 = 208 R=13$ \\
		 $208 : 16 = 13 R=0$ \\
		 $13 : 16 = 0 R=13$ \\
		 $\mathbf{?_{16} = \doubleunderline{D0D0}}$
	 }
\end{enumerate}

\section{Rechnungen}
F"uhren Sie die folgenden Rechnungen aus!

\begin{enumerate}
\item $77_{8} + B1D_{16} + 1100_{2} = ?_{10}$ \\ \\ %2920
	 {\color{ForestGreen}
	 	 $77_{8} = ?_{10}$ \\
		 $P(b)= 7 \cdot 8^0 + 7 \cdot 8^1$ \\
		 $P(b)= 7 + 56 $ \\
		 $\mathbf{P(8)= \doubleunderline{63}}$ \\
		 \\
	 	 $B1D_{16} = ?_{10}$ \\
		 $P(b)= 13 \cdot 16^0 + 1 \cdot 16^1 + 11 \cdot 16^2$ \\
		 $P(b)= 13 + 16 + 2816 $ \\
		 $\mathbf{P(16)= \doubleunderline{2845}}$ \\	
		 \\
	 	 $B1D_{2} = ?_{10}$ \\
		 $P(b)= 0 \cdot 2^0 + 0 \cdot 2^1 + 1 \cdot 2^2 + 1 \cdot 2^3$ \\
		 $P(b)= 0 + 0 + 4 + 8 $ \\
		 $\mathbf{P(2)= \doubleunderline{12}}$ \\		
		 \\
		 $\mathbf{Summe = 63 + 2845 + 12 = \doubleunderline{2920}}$ 
	 }
\item $11111111_{2} + 88_{10} = ?_{16}$  \\     \\    %343
	 {\color{ForestGreen}
	 	 $11111111_{2} = ?_{16}$ \\
		 $\underbrace{1111}_\text{$F_{16}$}\underbrace{1111}_\text{$F_{16}$}$ \\
		 $\mathbf{?_{16} = \doubleunderline{FF}}$ \\
		 \\
		 $88_{10} = ?_{16}$ \\
		 $88 : 16 = 5 R=8$  \\
		 $5 : 16 = 0 R=5$ \\
		 $\mathbf{?_{16} = \doubleunderline{58}}$ \\
		 \\
		 \textbf{\underline{Addition der beiden HEX Zahlen}} \\
		 \\
		 \begin{tabular}{c@{\,}c@{\,}c@{\,}c}
		  &  & $F$ & $F_{16}$ \\
		+ &   & $5$ & $8_{16}$ \\
		\hline
		"Ubertrag  & $1$ & $1$  &  \\
		\hline
		Summe & $1$ & $5$ & $7_{16}$ \\
		\hline
		\hline		
		\end{tabular}
		\vspace{4mm}
		\\
		 \textbf{\underline{Umwandlung ins 10er System}} \\
		 \\
	 	 $157_{16} = ?_{10}$ \\
		 $P(b)= 7 \cdot 16^0 + 5 \cdot 16^1 + 1 \cdot 16^2$ \\
		 $P(b)= 7 + 80 + 256 $ \\
		 $\mathbf{P(16)= \doubleunderline{343_{10}}}$ \\			 
	 }
\item $111_{2} + 222_{16} + 333_{8} = ?_{10}$ \\  \\   %772
	 {\color{ForestGreen}
	 	 $222_{16} = ?_{2}$ \\
		 $\underbrace{2}_\text{$0010_{2}$}\underbrace{2}_\text{$0010_{16}$}\underbrace{2}_\text{$0010_{16}$}$ \\
		 $\mathbf{?_{2} = \doubleunderline{1000100010}}$ \\
		 \\
	 	 $333_{8} = ?_{2}$ \\
		 $\underbrace{3}_\text{$011_{2}$}\underbrace{3}_\text{$011_{16}$}\underbrace{3}_\text{$011_{16}$}$ \\
		 $\mathbf{?_{2} = \doubleunderline{11011011}}$ \\
		 \\
		 \textbf{\underline{Addition der ersten beiden bin"aren Zahlen}} \\
		 \\
		 \begin{tabular}{c@{\,}c@{\,}c@{\,}c@{\,}c@{\,}c@{\,}c@{\,}c@{\,}c@{\,}c@{\,}c}
		&  &  &  &  &  &  &  &  $1$ & $1$ & $1_{2}$ \\
		+ & $1$ & $0$ & $0$ & $0$ & $1$& $0$ & $0$ & $0$ & $1$ & $0_{2}$ \\
		\hline
		"Ubertrag  & & & & & & & $1$ & $1$ &  &  \\
		\hline
		Summe & $1$ & $0$ & $0$ & $0$ & $1$ & $0$ & $1$ & $0$ & $0$ & $1_{2}$ \\
		\hline
		\hline		
		\end{tabular}
		\\
		\vspace{4mm}
		\\
		 \textbf{\underline{Addition der n"achsten bin"aren Zahl zum obigen Resultat}} \\
		 \\
		 \begin{tabular}{c@{\,}c@{\,}c@{\,}c@{\,}c@{\,}c@{\,}c@{\,}c@{\,}c@{\,}c@{\,}c}
		& $1$ & $0$ & $0$ & $0$ & $1$ & $0$ & $1$ & $0$ & $0$ & $1_{2}$ \\
		+ & & & $1$ & $1$ & $0$& $1$ & $1$ & $0$ & $1$ & $1_{2}$ \\
		\hline
		"Ubertrag  & & $1$ & $1$ & $1$ & $1$ & $1$ & & $1$ & $1$  &  \\
		\hline
		Summe & $1$ & $1$ & $0$ & $0$ & $0$ & $0$ & $0$ & $1$ & $0$ & $0_{2}$ \\
		\hline
		\hline		
		\end{tabular}
		\vspace{4mm}		
		\\
		 \textbf{\underline{Umwandlung ins 10er System}} \\
		 \\
	 	 $1100000100_{2} = ?_{10}$ \\
		 $P(b)= 0 \cdot 2^0 + 0 \cdot 2^1 + 1 \cdot 2^2 + 0 \cdot 2^3 + 0 \cdot 2^4 + 0 \cdot 2^5 + 0 \cdot 2^6 + 0 \cdot 2^7 + 1 \cdot 2^8 + 1 \cdot 2^9$ \\
		 $P(b)= 0 \cdot 1 + 0 \cdot 2 + 1 \cdot 4 + 0 \cdot 8 + 0 \cdot 16 + 0 \cdot 32 + 0 \cdot 64 + 0 \cdot 128 + 1 \cdot 256 + 1 \cdot 512$ \\
		 $P(b)= 4 + 256 + 512 $ \\
		 $\mathbf{P(16)= \doubleunderline{772_{10}}}$ \\			 
	 }

\end{enumerate}


\section{Fragen}

Beantworten Sie die folgenden Fragen!

\begin{enumerate}
\item Welches ist die kleinste Basis die ein Zahlensystem sinnvollerweise haben kann ? \\
	 {\color{ForestGreen}
	 	$\mathbf{b \geq 2}$ - Die Gr"osse b des Ziffernvorrats spielt eine entscheidende Rolle. Bei den wichtigen ganzzahligen Systemen ist der Wert der dargestellten Zahl die Summe der Produkte des Werts der Ziffer mit ihrem Stellenwert, 
	 	also ein Polynom in b mit den Werten der Ziffern als Koeffizienten - ergo bei einem un"aren System w"are die Stelle $1 (1^1)$ gleichwertig mit der Stelle $5 (1^5)$.
	}
\item Welche Unterschiede lassen sich im Bezug zur r"omischen Zahlenschrift benennen ? \\
	 {\color{ForestGreen}
		R"omische Ziffern haben einen festen Wert egal an welcher Stelle diese stehen.
		\\
		\textbf{z.B.} \verb|XXX| - sowohl die vorderste als auch die hinterste Ziffer hat den Wert 10 - beim dezimalen Zahlensystem h"atte die erste Ziffer den hundertfachen Wert im Vergleich zur hintersten Ziffer.
		\\
		Abgesehen von der subraktiven Schreibung ist der Wert unabh"angig von der Position und die Ziffern werden einfach aufaddiert.
	}
\item Warum haben die kalifornischen Yuki Indianer in Ihrer Sprache nur 8 Ziffern ? \\
	 {\color{ForestGreen}
		Die Indianer haben anstelle der Finger die Gruben zwischen den Fingern zum z"ahlen benutzt.
	}
\end{enumerate}
\end{document}