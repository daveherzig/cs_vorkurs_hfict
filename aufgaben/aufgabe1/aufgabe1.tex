\documentclass[a4paper,10pt]{article}
\topmargin-1cm
\addtolength{\textheight}{2.5cm}
\addtolength{\textwidth}{2cm}
\usepackage{times}

\usepackage{verbatim}
\usepackage[dvips]{graphicx}
\usepackage[german]{babel}
\usepackage[latin1]{inputenc}

\setlength{\parindent}{0cm}

\title{Latex Template}
\author{David Herzig}
\date{WS2003}

\begin{document}

HF-ICT - H"ohere Fachschule f"ur Informations- und Kommunikationstechnologie\\
CH 4132 Muttenz\\
D.Herzig

\vspace{2mm}

\begin{center}
{\Large \bf Vorkurs Informatik}\\
Exercise sheet 1
\end{center}

\vspace{2mm}

\line(1,0){400}

\vspace{5mm}

\section{Zahlensysteme}

F"uhren Sie die folgenden Umwandlungen durch!

\begin{enumerate}
\item $1010110111_{2} = ?_{10}$
\item $25_{16} = ?_{2}$
\item $2345_{8} = ?_{10}$
\item $ABCD_{16} = ?_{2}$
\item $1046_{10} = ?_{16}$
\end{enumerate}

\section{Rechnungen}
F"uhren Sie die folgenden Rechnungen aus!

\begin{enumerate}
\item $12_{10} + ABC_{16} + 1001_{2} = ?_{10}$
\item $10011101_{2} + 123_{10} = ?_{16}$
\item $11_{2} + 77_{8} + FF_{16} = ?_{10}$
\end{enumerate}


\section{Fragen}

Beantworten Sie die folgenden Fragen!

\begin{enumerate}
\item Wieviele Zahlensysteme gibt es?
\item Durch was kann ein Zahlensystem charakterisiert werden?
\item Wie k"onnte eine direkte Umwandlung vom Bin"arsystem ins Oktalsystem erfolgen?
\item Wieviele Stellen werden ben"otigt um eine Zahl Z mit Basis B darzustellen?
\item "Uberlegen Sie sich, wie man negative Zahlen im Bin"arsystem darstellen k"onnte.
\item Wenn die Menschen mit Ihren Fingern bin"ar z"ahlen w"urden (Finger ausgestreckt = 1, Finger eingezogen = 0), wie hoch k"onnten wir dann z"ahlen?
\end{enumerate}
\end{document}

