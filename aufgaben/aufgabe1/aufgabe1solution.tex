\documentclass[a4paper,10pt]{article}
\topmargin-1cm
\addtolength{\textheight}{2.5cm}
\addtolength{\textwidth}{2cm}
\usepackage{times}

\usepackage{verbatim}
\usepackage[dvips]{graphicx}
\usepackage[german]{babel}
\usepackage[latin1]{inputenc}

\setlength{\parindent}{0cm}

\title{Latex Template}
\author{David Herzig}
\date{WS2003}

\begin{document}

KTSI - Kantonale Technikerschule f"ur Informatik\\
CH 4132 Muttenz\\
D.Herzig

\vspace{2mm}

\begin{center}
{\Large \bf Vorkurs Informatik}\\
Exercise sheet 1 - Musterl"osung
\end{center}

\vspace{2mm}

\line(1,0){400}

\vspace{5mm}

\section{Zahlensysteme}

F"uhren Sie die folgenden Umwandlungen durch!

\begin{enumerate}
\item $1010110111_{2} = 695_{10}$
\item $25_{16} = 100101_{2}$
\item $2345_{8} = 1253_{10}$
\item $ABCD_{16} = 1010101111001101_{2}$
\item $1046_{10} = 416_{16}$
\end{enumerate}

\section{Rechnungen}
F"uhren Sie die folgenden Rechnungen aus!

\begin{enumerate}
\item $12_{10} + ABC_{16} + 1001_{2} = 2769_{10}$
\item $10011101_{2} + 123_{10} = 118_{16}$
\item $11_{2} + 77_{8} + FF_{16} = 321_{10}$
\end{enumerate}


\section{Fragen}

Beantworten Sie die folgenden Fragen!

\begin{enumerate}
\item Wieviele Zahlensysteme gibt es?\\
\emph{Unendlich viele}
\item Durch was kann ein Zahlensystem charakterisiert werden?\\
\emph{Basis, zur Verf"ugung stehende Ziffern, Stellenwert}
\item Wie k"onnte eine direkte Umwandlung vom Bin"arsystem ins Oktalsystem erfolgen?\\
\emph{Die bin"are Zahl wird in dreier Guppen aufgeteilt.}
\item Wieviele Stellen werden ben"otigt um eine Zahl Z mit Basis B darzustellen?\\
$\lfloor log_B(Z) + 1 \rfloor$
\item "Uberlegen Sie sich, wie man negative Zahlen im Bin"arsystem darstellen k"onnte.\\
\emph{Es wird ein Bit als Vorzeichen verwendet. Bsp: 1 bedeutet Negativ, 0 bedeutet Positiv.}
\item Wenn die Menschen mit Ihren Fingern bin"ar z"ahlen w"urden (Finger ausgestreckt = 1, Finger eingezogen = 0), wie hoch k"onnten wir dann z"ahlen?\\
\emph{Von 0 bis} $2^{10}-1 = 1023$
\end{enumerate}
\end{document}

