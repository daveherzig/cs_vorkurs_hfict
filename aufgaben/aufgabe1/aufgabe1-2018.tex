\documentclass[a4paper,10pt]{article}
\topmargin-1cm
\addtolength{\textheight}{2.5cm}
\addtolength{\textwidth}{2cm}
\usepackage{times}

\usepackage{verbatim}
\usepackage[dvips]{graphicx}
\usepackage[german]{babel}
\usepackage[latin1]{inputenc}

\setlength{\parindent}{0cm}

\title{Latex Template}
\author{David Herzig}
\date{WS2003}

\begin{document}

HF-ICT - H"ohere Fachschule f"ur Informations- und Kommunikationstechnologie\\
CH 4132 Muttenz\\
P.S"utterlin

\vspace{2mm}

\begin{center}
{\Large \bf Vorkurs Informatik}\\
Exercise sheet 1 - 2018 Edition
\end{center}

\vspace{2mm}

\line(1,0){400}

\vspace{5mm}

\section{Zahlensysteme}

F"uhren Sie die folgenden Umwandlungen durch!

\begin{enumerate}
\item $11110000_{2} = ?_{10}$     %240
\item $2AA_{16} = ?_{2}$                %1010101010
\item $765_{8} = ?_{10}$                %501
\item $C0DE_{16} = ?_{2}$             %1100000011011110
\item $53456_{10} = ?_{16}$         %D0D0
\end{enumerate}

\section{Rechnungen}
F"uhren Sie die folgenden Rechnungen aus!

\begin{enumerate}
\item $77_{8} + B1D_{16} + 1100_{2} = ?_{10}$   %2920
\item $11111111_{2} + 88_{10} = ?_{16}$           %343
\item $111_{2} + 222_{16} + 333_{8} = ?_{10}$     %1404
\end{enumerate}


\section{Fragen}

Beantworten Sie die folgenden Fragen!

\begin{enumerate}
\item Welches ist die kleinste Basis die ein Zahlensystem sinnvollerweise haben kann ?   %2 - Die Größe b des Ziffernvorrats spielt eine entscheidende Rolle. Bei den wichtigen ganzzahligen Systemen ist der Wert der dargestellten Zahl die Summe der Produkte des Werts der Ziffer mit ihrem Stellenwert, also ein Polynom in b mit den Werten der Ziffern als Koeffizienten - ergo bei einem unären System wäre die Stelle 1 (1^1) gleichwertig mit der Stelle 5 (1^5).
\item Welche Unterschiede lassen sich im Bezug zur r"omischen Zahlenschrift benennen ? %Römische Zahlen haben einen festen Wert egal an welcher Stelle diese stehen - abgesehen von der Subraktiven Schreibung ist der Wert unabhängig von der Position und wird einfach aufaddiert.
\item Warum haben die kalifornischen Yuki Indianer in Ihrer Sprache nur 8 Ziffern ? %Die Indianer haben anstelle der Finger die Gruben zwischen den Fingern zum zählen benutzt.
\end{enumerate}
\end{document}