\documentclass[a4paper,10pt]{article}
\topmargin-1cm
\addtolength{\textheight}{2.5cm}
\addtolength{\textwidth}{2cm}
\usepackage{times}

\usepackage{verbatim}
\usepackage[dvips]{graphicx}
\usepackage[german]{babel}
\usepackage[latin1]{inputenc}

\setlength{\parindent}{0cm}

\title{Latex Template}
\author{David Herzig}
\date{WS2003}

\begin{document}

HF-ICT - H"ohere Fachschule f"ur Informations- und Kommunikationstechnologie\\
CH 4132 Muttenz\\
D.Herzig

\vspace{2mm}

\begin{center}
{\Large \bf Vorkurs Informatik}\\
Exercise sheet 7
\end{center}

\vspace{2mm}

\line(1,0){400}

\vspace{5mm}

\section{Kleinste Zahl}
Schreiben Sie ein Programm, welches 10 Zahlen einliest und die kleinste dieser 10 Zahlen wieder ausgibt.

\section{Gr"osster gemeinsamer Teiler}
Erstellen Sie ein Programm, welches 2 nat"urliche Zahlen einliest und den ggt dieser beiden Zahlen berechnet.\\
Zur Berechnung kann der folgende Algorithmus verwendet werden:\\
Die kleiner Zahl der beiden Zahlen wird von der gr"osseren Zahl subtrahiert. Dies wird solange gemacht,
bis beide Zahlen gleich sind. Dies ist dann der ggt der beiden urspr"unglichen Zahlen.

\vspace{3mm}
Beispiel\\
a = 30, b = 54

\vspace{3mm}
b = 54 - 30 (Gr"ossere Zahl minus kleinere Zahl)\\
a = 30, b = 24\\
a = 30 - 24 (Gr"ossere Zahl minus kleinere Zahl)\\
a = 6, b = 24\\
b = 24 - 6 (Gr"ossere Zahl minus kleinere Zahl)\\
a = 6, b = 18\\
b = 18 - 6 (Gr"ossere Zahl minus kleinere Zahl)\\
a = 6, b = 12\\
b = 12 - 6 (Gr"ossere Zahl minus kleinere Zahl)\\
a = 6, b = 6\\
Beide Zahlen sind nun gleich. Der ggt von 30 und 54 ist 6.

\section{Primzahlen}
Schreiben Sie ein Programm, welches Primzahlen berechnet. Das Programm soll in einer endlos Schleife laufen
und nacheinander Primzahlen ausgeben.

\vspace{5mm}
2\\
3\\
5\\
7\\
11\\
13\\
...

\end{document}

