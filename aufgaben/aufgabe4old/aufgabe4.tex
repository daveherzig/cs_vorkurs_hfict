\documentclass[a4paper,10pt]{article}
\topmargin-1cm
\addtolength{\textheight}{2.5cm}
\addtolength{\textwidth}{2cm}
\usepackage{times}

\usepackage{verbatim}
\usepackage[dvips]{graphicx}
\usepackage[german]{babel}
\usepackage[latin1]{inputenc}

\setlength{\parindent}{0cm}

\title{Latex Template}
\author{David Herzig}
\date{WS2003}

\begin{document}

KTSI - Kantonale Technikerschule f"ur Informatik\\
CH 4132 Muttenz\\
D.Herzig

\vspace{2mm}

\begin{center}
{\Large \bf Vorkurs Informatik}\\
Exercise sheet 4
\end{center}

\vspace{2mm}

\line(1,0){400}

\vspace{5mm}

\section{Speicher}
Erkl"aren Sie den Begriff fl"uchtiger und nicht fl"uchtiger Speicher und nennen Sie je zwei Beispiele.

\section{Arbeitsspeicher und Steuerbus}
Ein Computer arbeitet mit einem Intel 8086 Prozessor und besitzt 20 Adressleitungen. Wie gross kann der Hauptspeicher in diesem Computer maximal sein?\\
{\small Hinweis: Schauen Sie im Internet nach den technischen Daten das Intel 8086 Prozessor)

\section{Festplatte und CPU}
Ein Computer hat einen Bus mit einer Zykluszeit von 25ns, in der er ein 32Bit Wort aus dem Speicher lesen oder schreiben kann. Der Computer besitzt auch eine SCSI Festplatte, welche den Bus benutzt und in 40MB/s l"auft. Die CPU ruft normalerweise alle 25ns eine 32Bit Instruktion ab und f"uhrt sie aus. Wie stark wird die CPU von der Festplatte gebremst?

\section{Filmkomprimierung}
Um einen 133-Minuten Film auf einer einseitigen ein Schicht DVD zu speichern, ist ein nicht unbetr"achtlicher Komprimierungsaufwand n"otig. Berechnen Sie den Komprimierungsfaktor. Gehen Sie davon aus, dass f"ur die Videospur 3.5GB zur Verf"ugung stehen, dass die Bildaufl"osung 720x420 Pixel mit 24Bit Farbtiefe betr"agt und die Bilder mit 30 Bildrahmen pro Sekunde dargestellt werden.

\end{document}

