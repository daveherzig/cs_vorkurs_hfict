\documentclass[a4paper,10pt]{article}
\topmargin-1cm
\addtolength{\textheight}{2.5cm}
\addtolength{\textwidth}{2cm}
\usepackage{times}

\usepackage{verbatim}
\usepackage[dvips]{graphicx}
\usepackage[german]{babel}
\usepackage[latin1]{inputenc}

\setlength{\parindent}{0cm}

\title{Latex Template}
\author{David Herzig}
\date{WS2003}

\begin{document}

HF-ICT - H"ohere Fachschule f"ur Informations- und Kommunikationstechnologie\\
CH 4132 Muttenz\\
D.Herzig

\vspace{2mm}

\begin{center}
{\Large \bf Vorkurs Informatik}\\
Exercise sheet 5
\end{center}

\vspace{2mm}

\line(1,0){400}

\vspace{5mm}

\section{Umlaufzeit eines Satelliten}
F"ur die Berechnung der Umlaufzeit eines Satelliten auf der Kreisbahn um die
Erde kann die folgende Formel verwendet werden:

\begin{displaymath}
T[sec] = \frac{2 \cdot \pi}{R_{E}} \cdot \sqrt{\frac{(R_{E}+h)^{3}}{g}}
\end{displaymath}

$g$ ist die Erdbeschleunigung: $g = 9.80665 m/s^{2}$\\
$R_{E}$ ist der Erdradius: $R_{E} = 6371 km$\\
$\pi = 3.14159$\\

Schreiben Sie ein Programm, das die H"ohe h des Satelliten einliest und dessen
Umlaufzeit ausgibt.\\
Wie hoch muss ein Satellit sein, damit er geostation"ar (Umlaufzeit = 24h) ist?


\section{Freier Fall}
Beim freien Fall befindet sich ein K"orper zun"achst in Ruhe und bewegt sich
unter dem Einfluss der Erdanziehung aus einer bestimmten H"ohe $h_0$ nach unten. Die
Dauer eines Falles kann mit der folgenden Formel berechnet werden:

\begin{displaymath}
T [sec] = \sqrt{\frac{2h_0}{g}}
\end{displaymath}

\emph{g} ist die Erdbeschleuningungskonstante. Ihr Wert betr"agt:\\
$g=9.80665$\\
\\
Schreiben Sie ein Programm, das die H"ohe $h_0$ einliest und den Wert der
Dauer des Falles ausgibt.


\section{Gleichf"ormige Bewegung}
Mit einer neuen Radaranlage misst die Polizei die Dauer, welche ein Fahrzeug f"ur eine bestimmte
Strecke ben"otigt. Daraus kann die Geschwindigkeit mit der folgenden Formel berechnet werden:

\begin{displaymath}
v [m/s] = \frac{s[m]}{t[s]}
\end{displaymath}

Schreiben Sie ein Programm, welches die Strecke sowie die Zeit einliest. Danach wird die Geschwindigkeit
berechnet und ausgegeben.

\end{document}

