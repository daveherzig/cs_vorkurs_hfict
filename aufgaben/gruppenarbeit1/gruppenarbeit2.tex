\documentclass[a4paper,10pt]{article}
\topmargin-1cm
\addtolength{\textheight}{2.5cm}
\addtolength{\textwidth}{2cm}
\usepackage{times}
\usepackage{hyperref}
\usepackage{verbatim}
\usepackage[dvips]{graphicx}
\usepackage[german]{babel}
\usepackage[latin1]{inputenc}
\usepackage{hyperref}

\setlength{\parindent}{0cm}

\title{Latex Template}
\author{Patrik S"utterlin}
\date{WS2003}

\begin{document}

HF-ICT - H"ohere Fachschule f"ur Informations- und Kommunikationstechnologie\\
CH 4132 Muttenz\\
Patrik S"utterlin

\vspace{2mm}

\begin{center}
{\Large \bf Vorkurs Informatik}\\
Tasks - Group 2
\end{center}

\vspace{2mm}

\line(1,0){400}

\vspace{5mm}

\section{''for-Schleifen'' Tutorial}
Erarbeiten Sie in Ihrer Gruppe folgende Aufgaben:

\begin{itemize}
\item \textbf{[10 Minuten]} Nehmen Sie sich Zeit um die Grundlagen der ''for Schleife'' zu verstehen (anhand der Hilfsmittel unten).
\item \textbf{[10 Minuten]} Implementieren Sie eine einfache ''for Schleife'' welche von 1 bis zu 10 z"ahlt. Bei jedem Schleifendurchgang soll der aktuelle Wert auf die Konsole ausgegeben werden.
\item \textbf{[10 Minuten]}  Schreiben Sie ein neues Aufgabenblatt f"ur Ihre Kurskollegen. "Uberlegen Sie sich eine einfache C++ Aufgabe welche als elementares Element eine ''for Schleife'' enthalten soll. W"ahlen sie den Aufgabentext so, dass diese als Aufgabe bis zum n"achsten Kurs selbsst"andig erarbeitet werden kann (Code + Diagramm).
\item \textbf{[10 Minuten Vorbereitungszeit f"ur die Pr"asentation]} In der zweiten Kurslektion erkl"aren Sie der anderen Gruppe wie die ''for Schleife'' funktioniert anhand der einfachen ''inkrementellen Z"ahlschleife'' (z"ahlen von 1 bis 10) - "uberlegen Sie sich wie sie dies am geschicktesten machen damit Ihre Kollegen Ihre Erkl"arungen nachvollziehen k"onnen. Im Anschluss erl"autern Sie die von Ihnen erstellte Aufgabe.
\item \textbf{[5 Minuten] Pause}
\item \textbf{[20 Minuten]} Pr"asentation Gruppe 1
\item \textbf{[20 Minuten]} Pr"asentation Gruppe 2
\end{itemize}

\textbf{Hilfsmittel:} 

\begin{itemize}
\item Kurs-Handout: \href{https://share.hf-ict.ch/index.php/apps/files/?dir=/HF-ICT/VKI_Vorkurs%20Informatik/unterlagen&fileid=114518}{share.hf-ict.ch $>$ VKI\_Vorkurs Informatik $>$ unterlagen $>$ 02\_grundlagen\_cpp} - ab Seite 32
\item Kurs-Handout: \href{https://share.hf-ict.ch/index.php/apps/files/?dir=/HF-ICT/VKI_Vorkurs%20Informatik/videos&fileid=114517}{share.hf-ict.ch $>$ VKI\_Vorkurs Informatik $>$ videos $>$ 06\_cpp\_loops} ab 07:25
\end{itemize}

\end{document}

