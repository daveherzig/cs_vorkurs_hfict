\documentclass[a4paper,10pt]{article}
\topmargin-1cm
\addtolength{\textheight}{2.5cm}
\addtolength{\textwidth}{2cm}
\usepackage{times}
\usepackage{hyperref}
\usepackage{verbatim}
\usepackage[dvips]{graphicx}
\usepackage[german]{babel}
\usepackage[latin1]{inputenc}
\usepackage{hyperref}

\setlength{\parindent}{0cm}

\title{Latex Template}
\author{Patrik S"utterlin}
\date{WS2003}

\begin{document}

HF-ICT - H"ohere Fachschule f"ur Informations- und Kommunikationstechnologie\\
CH 4132 Muttenz\\
Patrik S"utterlin

\vspace{2mm}

\begin{center}
{\Large \bf Vorkurs Informatik}\\
Tasks - Group 1
\end{center}

\vspace{2mm}

\line(1,0){400}

\vspace{5mm}

\section{hf-ict Taschenrechner ''Extended''}
Erarbeiten Sie in Ihrer Gruppe folgende Aufgaben:

\begin{itemize}
\item \textbf{[5 Minuten]} Diskutieren sie in der Gruppe welche Funktionen im Taschenrechner aus dem "Ubungsblatt 6 noch fehlen im Vergleich zu einem gew"ohnlichen handels"ublichen Taschenrechner. Definieren Sie auf der Basis Ihrer Diskussion zwei zus"atzliche Funktionen, welche mit dem aktuellen C++ Wissensstand der Kursteilnehmer im bestehenden Code noch zus"atzlich implementiert werden k"onnen.
\item \textbf{[15 Minuten]} Am letzten Kurstag haben Sie die Aufgabe erhalten ein Nassi-Schneiderman Diagramm f"ur den Taschenrechner zu erstellen. Passen Sie nun eben dieses Nassi-Schneiderman Diagramm so an, dass die beiden neuen Funktionen enthalten sind. Implementieren Sie dabei die ''switch'' Verzweigung und die ''do while'' Schleife im Diagramm (grafisch). 
\item \textbf{[10 Minuten]} Schreiben Sie ein neues Aufgabenblatt f"ur Ihre Kurskollegen in welchem sie die beiden geplanten Erweiterungen m"oglichst pr"azise umschrieben. W"ahlen sie den Aufgabentext so, dass diese als Aufgabe bis zum n"achsten Kurs selbsst"andig erarbeitet werden kann (...die Aufgabe umfasst den Code und das Diagramm).
\item \textbf{[10 Minuten Vorbereitungszeit f"ur die Pr"asentation]} In der zweiten Kurslektion pr"asentieren Sie als Team die Idee der anderen Gruppe. Sie erkl"aren die Aufgabe anhand des Diagramms (Projektor f"ur A4 Papier / Beamer stehem bereit). Zeitrahmen 15-20 Minuten.
\item \textbf{[5 Minuten] Pause}
\item \textbf{[20 Minuten]} Pr"asentation Gruppe 1
\item \textbf{[20 Minuten]} Pr"asentation Gruppe 2
\end{itemize}

\textbf{Hilfsmittel:} 

\begin{itemize}
\item Handout zu Nassi-Schneiderman Diagrammen von IBM 
\item Webpage: \href{http://en.cppreference.com/w/cpp/header/cmath}{C++ Referenz cmath.h}.
\end{itemize}

\end{document}

