\documentclass[a4paper,10pt]{article}
\topmargin-1cm
\addtolength{\textheight}{2.5cm}
\addtolength{\textwidth}{2cm}
\usepackage{times}

\usepackage{verbatim}
\usepackage[dvips]{graphicx}
\usepackage[german]{babel}
\usepackage[latin1]{inputenc}

\setlength{\parindent}{0cm}

\title{Latex Template}
\author{David Herzig}
\date{WS2004}

\begin{document}

KTSI - Kantonale Technikerschule f"ur Informatik\\
CH 4132 Muttenz\\
D.Herzig

\vspace{2mm}

\begin{center}
{\Large \bf Vorkurs Informatik}\\
Exercise sheet 4
\end{center}

\vspace{2mm}

\line(1,0){400}

\vspace{5mm}

\section{Einrichten}
Richten Sie sich zuhause (falls m"oglich) einen Arbeitsplatz ein, um selber C++ Programme entwerfen zu k"onnen.\\
Welche Programmieroberfl"ache Sie benutzen ist egal. Ich empfehle Ihnen KDevelop, welches unter SUsE Linux zur Ver"ugung steht.

\begin{itemize}
\item MinGW
\item Microsoft Visual Studio Express Edition (C++ .NET)
\item Linux
\item ...
\end{itemize}

\section{Hello World}
Zu Beginn bei Erlernen einer neuen Sprache ist es "ublich, dass ein Hello World Programm erstellt wird. Dieses gibt den einfachen Text \verb|"Hello World"| auf dem Bildschirm aus.

\section{Mathematische Ausdr"ucke}
Formulieren Sie jeden der folgenden mathematischen Ausdr"ucke als
C-Ausdruck. Verwenden Sie nur die minimale Anzahl Klammern zur Gruppierung von
Unterausdr"ucken. Nutzen Sie Assoziativit"aten und Priorit"aten der Operatoren
soweit m"oglich.

\begin{itemize}
\item $\frac{a}{b} - \frac{x}{y}$

\item $\frac{a+b}{a-b} - \frac{x-y}{x+y}$

\item $ax^{3}+bx^{2}+cx+d$

\item $\frac{1}{x}+\frac{2}{x^{2}}+\frac{3}{x^{3}}+\frac{4}{x^{4}}$
\end{itemize}


\section{Gr"ossen von Variablen}
Schreiben Sie ein Programm, dass die Speichergr"osse der folgenden Variablen ausgibt:

\begin{itemize}
\item float
\item double
\item int
\item long
\item char
\end{itemize}

\end{document}

