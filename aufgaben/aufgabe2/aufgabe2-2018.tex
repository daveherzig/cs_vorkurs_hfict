\documentclass[a4paper,10pt]{article}
\topmargin-1cm
\addtolength{\textheight}{2.5cm}
\addtolength{\textwidth}{2cm}
\usepackage{times}

\usepackage{verbatim}
\usepackage{graphicx}
\usepackage[german]{babel}
\usepackage[latin1]{inputenc}

\setlength{\parindent}{0cm}

\title{Latex Template}
\author{David Herzig}
\date{WS2004}

\begin{document}

HF-ICT - H"ohere Fachschule f"ur Informations- und Kommunikationstechnologie\\
CH 4132 Muttenz\\
P.S"utterlin

\vspace{2mm}

\begin{center}
{\Large \bf Vorkurs Informatik}\\
Exercise sheet 2 - 2018 Edition
\end{center}

\vspace{2mm}

\line(1,0){400}

\vspace{5mm}

\section{Gr"ossen der Informatik}

\begin{enumerate}

\item Auf Ihrer Festplatte befindet sich eine Datei \textbf{``foto.png''}. Im Eigenschaftenfenster der Datei sehen Sie, dass das File 18'220 Bytes an Nutzdaten umfasst - da Dateien aber immer aus einem oder mehreren Bl"ocken zu 4kB zusammengesetzt sind erscheint ausserdem eine separate Anzeige \textbf{``Gr"osse auf Datentr"ager''} ... d.h. der letzte Block wird von der Datei nicht immer vollst"andig belegt. \\
\\
Rechnen Sie aus:
% 18220 Bytes belegen 5 Bl"ocke zu 4096 Bytes - Ungenutzt um 5 Block sind (x = 18220 - 4 * 4096) 1836 Bytes
\begin{itemize}
\item Wieviele Bl"ocke die Datei belegen muss ?
\item Wieviel Platz die zugewiesenen Bl"ocke effektiv auf dem Datentr"ager belegen ?
\item Wieviele Bytes dabei ungenutzt bleiben ?
\end{itemize}

\item Wieviele bits werden ben"otigt, um die hexadezimale Zahl 88AA0244 zu speichern? \\
Wievielen Bytes entspricht das ? % 32 bit / 4 Bytes

\item Die Wellenform einer Musikdatei wird digital kodiert indem mehrere Amplitudenwerte einer Wellenform x mal in der Sekunde abgetastet werden. Der Wert der Amplitude wird mit einer Zahl fesgehalten - die Zahl kann dabei den Wert $0_{16}$ bis $FFFF_{16}$ annehmen - pro Sekunde der Aufzeichnung erfolgen 8000 Abtastungen (Aufl"osung = 8 kHz). Berechnen Sie nun die minimale "Ubertragungsrate (kbit/s) welche Ihr Internetprovider zur Verf"ugung stellen muss, damit ein Stereo Musikst"uck in Echtzeit "ubertragen werden kann (und zwar unkomprimiert) ? % Monodatenrate = 16 [bit] * 8000 [1/s] = 128kbit/s | Stereo = 256kbit/s

\vspace{3mm}

\begin{center}
\includegraphics[width=150pt]{Wave.png}
\end{center}

\end{enumerate}

\section{ASCII Code}

\begin{enumerate}
\item "Ubersetzen Sie die Hexziffern in lesbaren ASCII Text.\\
\verb|42 52 41 56 4F 21| %BRAVO!

\item Wie lautet der Bin"arcode der ASCII Zeichenfolge ? \\
\verb|HF-ICT|  %01001000 01000110 00101101 01001001 01000011 01010100

\item "Ubersetzen Sie den folgenden ASCII Text.\\
\verb|01000001 01110010 01110100 01110111 01101111 01110010 01101011| %Artwork
\end{enumerate}

\end{document}

