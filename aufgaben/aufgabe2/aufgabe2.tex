\documentclass[a4paper,10pt]{article}
\topmargin-1cm
\addtolength{\textheight}{2.5cm}
\addtolength{\textwidth}{2cm}
\usepackage{times}

\usepackage{verbatim}
\usepackage[dvips]{graphicx}
\usepackage[german]{babel}
\usepackage[latin1]{inputenc}

\setlength{\parindent}{0cm}

\title{Latex Template}
\author{David Herzig}
\date{WS2004}

\begin{document}

HF-ICT - H"ohere Fachschule f"ur Informations- und Kommunikationstechnologie\\
CH 4132 Muttenz\\
D.Herzig

\vspace{2mm}

\begin{center}
{\Large \bf Vorkurs Informatik}\\
Exercise sheet 2
\end{center}

\vspace{2mm}

\line(1,0){400}

\vspace{5mm}

\section{Gr"ossen der Informatik}

\begin{enumerate}

\item Ein Bildschirm hat eine Aufl"osung von 1024x768 Pixel. Jeder Pixel kann eine aus 256 verschiedenen Farben darstellen. Wie viel Speicher wird f"ur die Darstellung des ganzen Bildschirms ben"otigt?

\item Wieviele Bytes werden ben"otigt, um die hexadezimale Zahl ABCD zu speichern?

\item Ein Verlag m"ochte gerne ein bestimmtes Buch in elektronischer Form vertreiben. Das Buch besitzt im gesamten 768 Seiten und hat auf einer Seite minimal 1024 Zeichen und maximal 2048 Zeichen. F"ur die Speicherung eines Zeichens werden 2 Bytes ben"otigt.

\begin{itemize}
\item Wie viel Speicherplatz ben"otigt dieses Buch im Minimum?
\item Wie lange dauert es im Maximum, das Buch "uber eine Datenleitung mit einer Geschwindigkeit von 16 kBytes pro Sekunde zu "ubertragen?
\end{itemize}

\end{enumerate}

\section{ASCII Code}
\begin{enumerate}

\item "Ubersetzen Sie den folgenden ASCII Text.\\
\verb|41 4C 4C 45 53 20 4B 4C 41 52 3F|

\item Wie lautet der Bin"arcode des Buchstaben X?

\item "Ubersetzen Sie den folgenden ASCII Text.\\
\verb|01000011 01001100 01000001 01001001 01010010 01000101|

\end{enumerate}

\section{Menschliches Auge}
Das Sichtfeld eines menschlichen Auges besteht aus $10^6$ Pixeln.
Jeder Pixel besteht aus je einen Wert f�r Rot, Gr"un und Blau,
wobei jeder Wert eine von 64 Helligkeitsstufen aufweist. Die zeitliche
Aufl"osung betr�gt 100ms. Berechnen Sie die Speichergr"osse eines Bildes,
welches das Auge erfassen kann, sowie die Datenrate des Auges.

\section{What's App Messages}
Ein Bekannter von Ihnen hat ein Mobile Abo bei der Swisscom mit einem Datenvolumen von
250MB. In seiner Freizeit schreibt er haupts"achlich What's App Nachrichten. Seine 
Nachrichten sind im Durchschnitt 280 Zeichen lang, wobei jedes Zeichen 2 Bytes ben"otigt. Zus"atzlich wird f"ur jede Nachricht 64 Bytes zus"atzliche
Information (zb. Nachricht gesendet, Nachricht empfangen, Nachricht gelesen)
versendet.
Wieviele Nachrichten kann er pro Monat schreiben, ohne sein Datenvolumen zu "uberschreiten?

\end{document}

