\documentclass[a4paper,10pt]{article}
\topmargin-1cm
\addtolength{\textheight}{2.5cm}
\addtolength{\textwidth}{2cm}
\usepackage{times}

\usepackage{verbatim}
\usepackage[dvips]{graphicx}
\usepackage[german]{babel}
\usepackage[latin1]{inputenc}

\setlength{\parindent}{0cm}

\title{Latex Template}
\author{David Herzig}
\date{WS2003}

\begin{document}

HF-ICT - H"ohere Fachschule f"ur Informations- und Kommunikationstechnologie\\
CH 4132 Muttenz\\
D.Herzig

\vspace{2mm}

\begin{center}
{\Large \bf Vorkurs Informatik}\\
Exercise sheet 6
\end{center}

\vspace{2mm}

\line(1,0){400}

\vspace{5mm}

\section{Schaltjahr}
Schreiben Sie ein Programm, welches ein Jahr einliest und dann bestimmt, ob dieses
Jahr ein Schaltjahr ist oder nicht.\\
Um zu pr"ufen ob ein Jahr ein Schaltjahr ist, wird folgende Regel verwendet:
\begin{itemize}
\item Ist das eingegebene Jahr J ein Jahrhundertjahr (1900, 2000, 2100, ...), dann ist J genau dann ein Schaltjahr,
wenn es ohne Rest durch 400 teilbar ist.
\item Ist das eingegebene Jahr J kein Jahrhunderjahr, dann ist J genau dann ein Schaltjahr, wenn es ohne Rest
durch 4 teilbar ist.
\end{itemize}

\section{Detailh"andler}
Ein Detailh"andler gibt immer \verb|5%| Rabatt, wenn von einem bestimmten Produkt mindestens 10 St"uck gekauft
werden. Ab 100 St"uck wird sogar \verb|8%| Rabatt gegeben.\\
Schreiben Sie ein Programm, welches den St"uckpreis und die Menge einliest. Danach wird der Gesamtpreis berechnet
und ausgegeben.

\section{Taschenrechner}
Schreiben Sie ein Programm, welches 2 Zahlen sowie ein Operator (+, -, *, /) einliest und diese Rechnung dann ausf"uhrt.
Das Resultat wird dann auf der Konsole ausgegeben.\\

\verb|Zahl 1: 43|\\
\verb|Zahl 2: 8|\\
\verb|Operator: +|\\
\verb|Resultat: 51|

\section{Nassi-Schneiderman Diagramm}
Zeichnen Sie die Logik welche Sie in den Aufgaben 1-3 programmiert haben als Nassi-Schneiderman Diagramm. \\ 
\\
Kopieren Sie zudem den Code ihres Taschenrechners (siehe Aufgabe 3) in ein neues .cpp Quellfile und \"andern Sie den Code so ab, dass neu eine \textbf{switch()} Kontrollstruktur anstelle von \textbf{if()} Bedingungen verwendet wird. \\
\\
Zeichnen Sie nun das passende NS-Diagramm welches die switch() Kontrollstruktur darstellt - recherchieren Sie im Internet wie dies aussehen muss.

\end{document}

