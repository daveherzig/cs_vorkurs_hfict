\documentclass[a4paper,10pt]{article}
\topmargin-1cm
\addtolength{\textheight}{2.5cm}
\addtolength{\textwidth}{2cm}

\usepackage{verbatim}
\usepackage[german]{babel}
\usepackage[latin1]{inputenc}
\usepackage{struktex}

\setlength{\parindent}{0cm}

\title{Latex Template}
\author{Patrik S"utterlin}
\date{WS2018}

\begin{document}

HF-ICT - H"ohere Fachschule f"ur Informations- und Kommunikationstechnologie\\
CH 4132 Muttenz\\
Patrik S"utterlin

\vspace{2mm}

\begin{center}
{\Large \bf Vorkurs Informatik}\\
Exercise sheet 6 - Nassi-Schneiderman Diagrams
\end{center}

\vspace{2mm}
\line(1,0){400}
\vspace{5mm}


\begin{center}
\begin{struktogramm}(125,42)[Loesung 6-1 Variante 1: Schaltjahrbestimmung anhand einer Jahreszahl]
  \ifthenelse{4}{4}{Jahreszahl ist restfrei teilbar durch 100}{ja}{nein}
    \ifthenelse{2}{2}{Jahreszahl ist restfrei teilbar durch 400}{ja}{nein}
      \assign{Jahr ist ein Schaltjahr}
    \change
      \assign{Jahr ist kein Schaltjahr}
    \ifend
  \change
    \ifthenelse{2}{2}{Jahreszahl ist restfrei teilbar durch 4}{ja}{nein}
      \assign{Jahr ist ein Schaltjahr}
    \change
      \assign{Jahr ist kein Schaltjahr}
    \ifend
  \ifend
\end{struktogramm}
\end{center}
\line(1,0){400}
\vspace{4mm}

Daraus lassen sich folgende logischen Ausdr\"ucke ableiten:\\ \\
\textbf{p =} Schaltjahr \\
\textbf{q =} kein Schaltjahr \\
\\
\textbf{Variablen:} \\
\textbf{j =} Jahreszahl \\
\\
$\mathbf{p} = ((j \bmod{100} = 0) \land (j \bmod{400} = 0)) \lor ((j \bmod{100} \neq 0) \land (j \bmod 4 = 0))$ \\
$\mathbf{q} = ((j \bmod{100} = 0) \land (j \bmod{400} \neq 0)) \lor ((j \bmod{100} \neq 0) \land (j \bmod 4 \neq 0))$ 

\vspace{4mm}
\line(1,0){400}
\begin{center}
\begin{struktogramm}(125,75)[Loesung 6-1 Variante 2: Schaltjahrbestimmung anhand einer Jahreszahl]
  \ifthenelse{4}{1}{Jahreszahl ist restfrei teilbar durch 4}{ja}{nein}
    \ifthenelse{2}{4}{Jahreszahl ist restfrei teilbar durch 400}{ja}{nein}
      \assign{Jahr ist ein Schaltjahr}
    \change
       \ifthenelse{2}{2}{Jahreszahl ist restfrei teilbar durch 100}{ja}{nein}
         \assign{Jahr ist kein Schaltjahr}
       \change
         \assign{Jahr ist ein Schaltjahr}
       \ifend
    \ifend
  \change
      \assign{Jahr ist kein Schaltjahr}
  \ifend
\end{struktogramm}

\begin{struktogramm}(125,40)[Loesung 6-2: Rabatt je nach verkaufter Menge]
    \ifthenelse{5}{2}{Verkaufte Menge ist gr\"osser oder gleich 100}{ja}{nein}
      \ifthenelse{2}{4}{Verkaufte Mege ist gr\"osser oder gleich 10}{ja}{nein}
        \assign{Rabatt = 8\%}
      \change
        \assign{Rabatt = 5\%}
      \ifend
    \change
      \assign{Rabatt = 0\%}
    \ifend
\end{struktogramm}

\vspace{4mm}
\line(1,0){400}
\vspace{4mm}

\begin{struktogramm}(125,140)[Loesung 6-3: Taschenrechner if-case Solution]
  \assign
  {%
    \begin{declaration}[Local Variables:]
     \description{\pVar{z1}}{a \pKey{double} variable - Number located before the operator in the calculation}
     \description{\pVar{z2}}{a \pKey{double} variable - Number located the operator in the calculation}
     \description{\pVar{op}}{a \pKey{char} variable - A character that specifies the operator type used in the calculation (* / - +)}
     \description{\pVar{result}}{a \pKey{double} variable - Stores the result of the calculation}
    \end{declaration}
   }
    \assign{Prompt for Parameter z1}
    \assign{\(z1 \gets cin\)}
    \assign{Prompt for Operator op}
    \assign{\(op \gets cin\)}
    \assign{Prompt for Parameter z2}
    \assign{\(z2 \gets cin\)}
    
    \ifthenelse{5}{5}{\(op\) equals the char '+'}{yes}{no}
      \assign{\(result = z1 + z2\)}
    \change
      \assign{No Action}
    \ifend
    
    \ifthenelse{5}{5}{\(op\) equals the char '-'}{yes}{no}
      \assign{\(result = z1 - z2\)}
    \change
      \assign{No Action}
    \ifend         
        
    \ifthenelse{5}{5}{\(op\) equals the char '*'}{yes}{no}
      \assign{\(result = z1 * z2\)}
    \change
      \assign{No Action}
    \ifend

    \ifthenelse{5}{5}{\(op\) equals the char '/'}{yes}{no}
      \assign{\(result = z1 / z2\)}
    \change
      \assign{No Action}
    \ifend

    \assign{Print \(result\) to screen}
\end{struktogramm}

\begin{struktogramm}(125,140)[Loesung 6-3: Taschenrechner switch-case Solution + Loop]
  \assign
  {%
    \begin{declaration}[Local Variables:]
     \description{\pVar{z1}}{a \pKey{double} variable - Number located before the operator in the calculation}
     \description{\pVar{z2}}{a \pKey{double} variable - Number located the operator in the calculation}
     \description{\pVar{op}}{a \pKey{char} variable - A character that specifies the operator type used in the calculation (* / - +)}
     \description{\pVar{result}}{a \pKey{double} variable - Stores the result of the calculation}
     \description{\pVar{answer}}{a \pKey{char} variable - Stores the user decision (continue / exit)}
    \end{declaration}
   }
  \until{\(answer != 'x'\)}   
    \assign{Prompt for Parameter z1}
    \assign{\(z1 \gets cin\)}
    \assign{Prompt for Operator op}
    \assign{\(op \gets cin\)}
    \assign{Prompt for Parameter z2}
    \assign{\(z2 \gets cin\)}
    
    \case[20]{2}{4}{switch(\(op\))}{'+'}
   \assign{\(result = z1 + z2\)}
   \switch{'-'}
   \assign{\(result = z1 - z2\)}
   \switch{'*'}
   \assign{\(result = z1 * z2\)}
   \switch{'/'}
   \assign{\(result = z1 / z2\)}
   \caseend

    \assign{Print \(result\) to screen}
    \assign{Ask the user if another calculation shall be performed}
    \assign{\(repeat \gets cin\)}
  \untilend  
\end{struktogramm}
\end{center}
\end{document}