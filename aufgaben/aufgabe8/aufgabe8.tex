\documentclass[a4paper,10pt]{article}
\topmargin-1cm
\addtolength{\textheight}{2.5cm}
\addtolength{\textwidth}{2cm}
\usepackage{times}

\usepackage{verbatim}
\usepackage[dvips]{graphicx}
\usepackage[german]{babel}
\usepackage[latin1]{inputenc}

\setlength{\parindent}{0cm}

\title{Latex Template}
\author{David Herzig}
\date{WS2003}

\begin{document}

HF-ICT - H"ohere Fachschule f"ur Informations- und Kommunikationstechnologie\\
CH 4132 Muttenz\\
D.Herzig

\vspace{2mm}

\begin{center}
{\Large \bf Vorkurs Informatik}\\
Exercise sheet 8
\end{center}

\vspace{2mm}

\line(1,0){400}

\vspace{5mm}

\section{Dreieck}
Schreiben Sie ein Programm zur Ausgabe eines Dreiecks auf der Konsole. Zuerst wird die Anzahl Reihen eingelesen.

\vspace{5mm}

\verb|Anzahl Reihen: 6|\\
\verb|*|\\
\verb|**|\\
\verb|***|\\
\verb|****|\\
\verb|*****|\\
\verb|******|\\

\section{Roulette}
Schreiben Sie ein Programm zur Simulation eines Casino Besuches.

\vspace{3mm}

Der Besucher geht mit einem bestimmten Betrag \verb|m| (z.b. CHF 5000) ins Casino und spielt Roulette (beim Roulette gibt es die Zahlen 0-36).
Der Besucher setzt immer einen fixen Betrag (Einsatz) \verb|n| (z.b. CHF 20) auf eine gerade Zahl. D.h. wenn eine 0 oder eine ungerade Zahl
kommt verliert der Besucher sein Einsatz, falls eine gerade Zahl kommt, so gewinnt er den Einsatz.\\
Schreiben Sie ein Programm, welches die Werte \verb|m| und \verb|n| einliest. Danach wird das Roulette Spiel solange simuliert,
bis der Besucher etweder kein Geld mehr hat, oder seinen Einsatz \verb|m| verdoppelt hat. Geben Sie aus, an wievielen Roulette Spielen
der Besucher teilgenommen hat (z.b. Der Besucher hat den Betrag von CHF 10000 innerhalb von 105 Spielen verloren).


\end{document}

